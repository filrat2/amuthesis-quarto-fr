% This is a LaTeX thesis template for Adam Mickiewicz University.
% to be used with quarto
% This template was produced by Jakub Nowosad
% Version: 22 July 2023

% Inspired by:
% This is a LaTeX thesis template for Monash University.
% to be used with Rmarkdown
% This template was produced by Rob Hyndman
% Version: 6 September 2016

\documentclass{amuthesis}
% \usepackage[polish]{babel}
\usepackage{polski}
\renewcommand{\figurename}{Rycina} % Redefine default figure caption %
\renewcommand{\tablename}{Tabela} % Redefine default table caption %
%%%%%%%%%%%%%%%%%%%%%%%%%%%%%%%%%%%%%%%%%%%%%%%%%%%%%%%%%%%%%%%
% Add any LaTeX packages and other preamble here if required
%%%%%%%%%%%%%%%%%%%%%%%%%%%%%%%%%%%%%%%%%%%%%%%%%%%%%%%%%%%%%%%
\usepackage{booktabs,tabularx} % Allows kableExtra to work %
\usepackage{indentfirst} % Adds indent in the first paragraph %
\usepackage{bookmark} % Adds indent in the first paragraph %

\author{Filip Ratajszczak}
\title{Wykrywanie farm fotowoltaicznych na podstawie danych
teledetekcyjnych}
\def\titleeng{My title}
\def\degreetitle{Praca inżynierska}
\def\major{Geoinformacja}
\def\albumid{461791}
\def\thesisyear{2024}

% Add subject and keywords below
\hypersetup{
     %pdfsubject={The Subject},
     %pdfkeywords={Some Keywords},
     pdfauthor={Filip Ratajszczak},
     pdftitle={Wykrywanie farm fotowoltaicznych na podstawie danych
teledetekcyjnych},
     pdfproducer={quarto with LaTeX}
}

\bibliography{thesis.bib, packages.bib}

\begin{document}

\pagenumbering{arabic}

\titlepage

\bookmarksetup{startatroot}

\hypertarget{streszczenie}{%
\chapter*{Streszczenie}\label{streszczenie}}
\addcontentsline{toc}{chapter}{Streszczenie}

\markboth{Streszczenie}{Streszczenie}

\textbf{Abstrakt}

Streszczenie powinno przedstawiać skrótowo główny problem pracy i jego
rozwiązanie. Możliwa struktura streszczenia to: (1) 1-3 zdania wstępu do
problemu (czym się zajmujemy, dlaczego jest to ważne, jakie są
problemy/luki do wypełnienia), (2) 1 zdanie opisujące cel pracy, (3) 1-3
zdania przedstawiające użyte materiały (dane) i metody (techniki,
narzędzia), (4) 1-3 zdania obrazujące główne wyniki pracy, (5) 1-2
zdania podsumowujące; możliwe jest też określenie dalszych
kroków/planów.

Słowa kluczowe: (4-6 słów/zwrotów opisujących treść pracy, które nie
wystąpiły w tytule)

\textbf{Abstract}

The abstract must be consistent with the above text.

Keywords: (as stated before)

\newpage

\setstretch{1.2}\sf\tighttoc\doublespacing

\bookmarksetup{startatroot}

\hypertarget{sec-wprowadzenie}{%
\chapter{Wprowadzenie}\label{sec-wprowadzenie}}

Wprowadzenie powinno mieć charakter opisu od ogółu do szczegółu (np.
trzy-pięć paragrafów). Pierwszy paragraf powinien być najbardziej
ogólny, a kolejne powinny przybliżać czytelnika do problemu.
Przedostatni paragraf powinien określić jaki jest problem (są problemy),
który praca ma rozwiązać i dlaczego jest to (są one) ważne.

Wprowadzenie powinno być zakończone stwierdzeniem celu pracy. Dodatkowo
tutaj może znaleźć się również krótki opis co zostało zrealizowane w
pracy.

\bookmarksetup{startatroot}

\hypertarget{sec-lit}{%
\chapter{Przegląd literatury}\label{sec-lit}}

Ten rozdział zawiera wyjaśnienie kontekstu pracy.

Pisząc ten rozdział proszę pomyśleć o osobach, które zupełnie nie znają
opisywanej tematyki. Należy tutaj krok po kroku wyjaśnić podstawowe
koncepcje, istotność problemu, wyniki poprzednich podobnych badań, itd.
Ten rozdział obejmuje tylko kwestie, które już zostały wykonane przez
inne osoby - nowe wyniki mają swoje miejsce w rozdziale
\ref{sec-wyniki}.

Każda kwestia opisana w tym rozdziale powinna być cytowana. Dodatnie
cytowania odbywa się poprzez uzupełnienie pliku \texttt{thesis.bib}
zapisem w formacie BibTeX, a następnie dodanie nazwy referencji
poprzedzonej znakiem \texttt{@}. Przykładowo, zacytowanie książki
Geocomputation with R odbywa się poprzez
\autocite{lovelace_geocomputation_2019}.

W przypadku, gdy cytowanie zostało poprawnie wpisane oraz istnieje w
pliku \texttt{thesis.bib} to bibliografia powinna się automatycznie
wygenerować na końcu pracy.

W przypadku, gdy praca dyplomowa opisuje konkretny obszar to można po
tym rozdziale stworzyć kolejny rozdział opisujący ``obszar badań''.

Ten i kolejne rozdziału moją mieć także podrozdziały. Tworzenie
podrozdziałów polega na stworzeniu nowej linii rozpoczynającej się od
znaków \texttt{\#\#} a następnie tytułu podrozdziału. Dodatkowo w
postaci \texttt{\{\#sec-\}} można dodać skrót nazwy
rozdziału/podrozdziału umożliwiający odnoszenie się do niego używając
operatora \texttt{{[}-@sec{]}}.

\hypertarget{sec-podr}{%
\section{Podrozdział}\label{sec-podr}}

Przykładowo, ``te kwestie zostały opisane w podrozdziale
\ref{sec-podr}''. Zwróć uwagę, że w ten sposób automatycznie tworzony
jest odnośnik w pliku PDF.

\bookmarksetup{startatroot}

\hypertarget{sec-materialy}{%
\chapter{Materiały}\label{sec-materialy}}

\hypertarget{sec-satellite-imagery}{%
\section{Zdjęcia satelitarne}\label{sec-satellite-imagery}}

\hypertarget{sec-sentinel1}{%
\subsection{Sentinel-1}\label{sec-sentinel1}}

\hypertarget{sec-sentinel2}{%
\subsection{Senitnel-2}\label{sec-sentinel2}}

Misja Sentinel-2 stanowi inicjatywę Komisji Europejskiej, która jest
operacyjnie prowadzona przez Europejską Agencję Kosmiczną (ang.
\emph{European Space Agency}, ESA) w ramach programu Copernicus. Celem
tej misji jest dostarczanie obrazów satelitarnych, obejmujących
trzynaście zakresów spektralnych o różnych rozdzielczościach
przestrzennych: 10, 20 lub 60 metrów, zależnie od rejestrowanego kanału.
Rozdzielczość czasowa misji Sentinel-2 wynosi pięć dni nad równikiem i
zwiększa się wraz ze wzrostem szerokości geograficznej, osiągając dwa
dni na średnich szerokościach geograficznych
\autocite{hejmanowska_2020_dane}.

Dane pozyskiwane przez satelity Sentinel-2 są dostępne na różnych
poziomach przetworzenia, lecz najczęściej używane przy tworzeniu map
pokrycia terenu i użytkowania ziemi (ang. \emph{Land Use/Land Cover},
LULC) są produkty 1C (współczynnik odbicia na poziomie górnej części
atmosfery; ang. \emph{Top-of-Atmospheric reflectance}, TOA) oraz 2A
(współczynnik odbicia na powierzchni Ziemi; ang.
\emph{Bottom-of-Atmospheric reflectance}, BOA)
\autocite{phiri_2020_sentinel2}.

Produkty poziomu 1C to dane poddane korekcjom radiometrycznym i
geometrycznym, prezentowane jako sceny o powierzchni 100
km\textsuperscript{2} (100 x 100 km) w projekcji UTM/WGS84
\autocite{esa_2015_sentinel2handbook}. Skuteczne wykorzystanie tych
danych w zastosowaniach związanych z terenami lądowymi wymaga
precyzyjnej korekcji zdjęć satelitarnych pod kątem efektów
atmosferycznych \autocite{main-knorn_2017_Sen2Cor}. Produkty poziomu 2A
powstają poprzez zastosowanie dodatkowej korekcji atmosferycznej dla
danych poziomu 1C za pomocą procesora korekcji atmosferycznej Sen2Cor
\autocite{main-knorn_2017_Sen2Cor}.

\hypertarget{tbl-tabela1}{}
\begin{table}
\caption{\label{tbl-tabela1}Kanały spektralne satelitów Sentinel-2 }\tabularnewline

\centering
\begin{tabular}{>{\raggedright\arraybackslash}p{1.5cm}>{\raggedright\arraybackslash}p{4cm}>{\raggedleft\arraybackslash}p{2cm}>{\raggedright\arraybackslash}p{2cm}>{\raggedleft\arraybackslash}p{2cm}}
\toprule
Kanał & Nazwa kanału & Centralna długość fali [nm] & Zakres spektralny [nm] & Rozdzielczość przestrzenna [m]\\
\midrule
B01 & Coastal Aerosol & 443 & 433–453 & 60\\
B02 & Blue & 490 & 458–523 & 10\\
B03 & Green & 560 & 543–578 & 10\\
B04 & Red & 665 & 650–680 & 10\\
B05 & Vegetation RedEdge & 705 & 698–713 & 20\\
\addlinespace
B06 & Vegetation RedEdge & 740 & 733–748 & 20\\
B07 & Vegetation RedEdge & 783 & 773–793 & 20\\
B08 & NIR & 842 & 785–900 & 10\\
B8A & NIR & 865 & 855–875 & 20\\
B09 & Water Vapour & 945 & 935–955 & 60\\
\addlinespace
B10 & Cirrus & 1375 & 1360–1390 & 60\\
B11 & SWIR & 1610 & 1565–1655 & 20\\
B12 & SWIR & 2190 & 2100–2280 & 20\\
\bottomrule
\end{tabular}
\end{table}

Z dostępnych kanałów spektralnych (Tabela \ref{tbl-tabela1})
wykorzystano 10 zakresów, ponieważ pasma rejestrowane w rozdzielczości
60 metrów są przeznaczone głównie do korekcji atmosferycznych i detekcji
chmur. Kanał 1 (443 nm) służy do korekcji wpływu aerozoli, kanał 9 (940
nm) do korekcji wpływu pary wodnej, a kanał 10 (1375 nm) do wykrywania
chmur typu cirrus \autocite{drusch_2012_sen2GMES}.

\hypertarget{indeksy-spektralne--}{%
\subsection{Indeksy spektralne - ???}\label{indeksy-spektralne--}}

\hypertarget{tekstury-obrazu--}{%
\subsection{Tekstury obrazu - ???}\label{tekstury-obrazu--}}

\hypertarget{sec-mosaics}{%
\section{Ortofotomapa i mozaiki zdjęć satelitarnych}\label{sec-mosaics}}

Do lokalizacji oraz wektoryzacji istniejących farm fotowoltaicznych
wykorzystano ortofotomapę udostępnianą przez Główny Urząd Geodezji i
Kartografii oraz mozaiki zdjęć satelitarnych dostarczane przez podmioty
komercyjne. W teledetekcji jednym z zastosowań mozaik obrazów
satelitarnych jest tworzenie zestawów danych referencyjnych poprzez
interpretację wizualną, np. w celu walidacji wyników klasyfikacji
produktów pokrycia terenu \autocite{lesiv_2018_sat_imagery_mosaics}. Do
stworzenie zbioru danych testowych i treningowych wykorzystano
ortomozaiki Google Satellite, Bing Aerial, Mapbox Satellite oraz Planet
Basemaps, udostępniane w formie usług sieciowych (WMS, WMTS, XYZ Tiles).
Ortomozaiki te są tworzone na podstawie komercyjnych zdjęć satelitarnych
wykonywanych przez podmioty takie jak Maxar Technologies, Airbus czy
Planet Labs. Ortofotomapa udostępniana przez GUGiK charakteryzuje się
rozdzielczością przestrzenną 25 cm, a rozdzielczość przestrzenna
wykorzystanych mozaik obrazów satelitarnych (poza Planet Basemaps) jest
wyższa niż 1 m, np. dla mozaiki Bing Aerial udostępnianej przez firmę
Microsoft wynosi ona 30-60 cm. Często jednak nie jest możliwie ustalenie
dat wykonania zdjęć satelitarnych, które posłużyły do stworzenia
konkretnej mozaiki zobrazowań satelitarnych
\autocite{lesiv_2018_sat_imagery_mosaics}. Mozaika tworzona przez Planet
na podstawie zdjęć satelitarnych wykonywanych przez konstelację
satelitów PlanetScope charakteryzuje się rozdzielczością przestrzenną
4,77 m na równiku, jednak w porównaniu do pozostałych wymienionych
produktów jest tworzona z miesięczną oraz kwartalną częstotliwością.
Pozwala to, pomimo niższej rozdzielczości przestrzennej na stworzenie
zbioru danych testowych i treningowych na konkretny okres czasu
\autocite{planet-basemaps-product-specifications.pdf}. Mozaiki Planet
Basemaps są tworzone na podstawie zdjęć wybieranych przy użyciu
algorytmu, który wybiera najlepsze obrazy z katalogu Planet w określonym
przedziale czasowym. Wybierając najlepsze obrazy, Planet jest w stanie
tworzyć wysokorodzielcze mozaiki, które są dokładne radiometrycznie i
przestrzennie, a także charakteryzują się zminimalizowanym wpływem
czynników atmosferycznych
\autocite[??]{planet-basemaps-product-specifications.pdf}.

?Rycina przedstawiająca jedną farmę na ortofotomapie, mozaikach
satelitarnych i scenie Sentinel-2 (kompozycja RGB)

\hypertarget{sec-samples-materials}{%
\section{Próbki treningowe i testowe}\label{sec-samples-materials}}

\bookmarksetup{startatroot}

\hypertarget{sec-metody}{%
\chapter{Metody}\label{sec-metody}}

\hypertarget{sec-processing}{%
\section{Przygotowanie danych}\label{sec-processing}}

Resampling kanałów w rozdzielczości 20 m do 10 m (metoda=bilinear).
Złączenie kanałów w wielokanałowy raster. Opisać pozostałe czynności
wykonane z tymi danymi.

?rycina - schemat przygotowania danych ?rycina - workflow Sentinel-1

\hypertarget{sec-samples-methods}{%
\section{Próbki treningowe i testowe}\label{sec-samples-methods}}

\hypertarget{sec-machine-learning}{%
\section{Uczenie maszynowe}\label{sec-machine-learning}}

Klasyfikacja obrazów w teledetekcji polega na grupowaniu komórek w
niewielkie zestawy klas, aby komórki w tych samych klasach miały podobne
właściwości \autocite{ismail_2009_classification}. Istnieje wiele
różnych metod klasyfikacji danych teledetekcyjnych. Stosunkowo nowymi
podejściami wykorzystywanymi w tym kontekście są metody oparte na
sztucznej inteligencji, takie jak uczenie maszynowe (ang. \emph{Machine
Learning}, ML) lub uczenie głębokie (ang. \emph{Deep Learning}, DL)
\autocite{hejmanowska_2020_dane}.

Uczenie maszynowe stanowi obszar sztucznej inteligencji, koncentrujący
się na opracowywaniu algorytmów i modeli statystycznych zapewniających
systemom komputerowym możliwość automatycznego uczenia się z danych i
wykonywania określonych zadań bez konieczności bezpośredniego
programowania. W przypadku skomplikowanych i złożonych zestawów danych
nie jesteśmy w stanie odpowiednio ich zinterpretować oraz wydobyć
poprawnych informacji po wizualnym przejrzeniu danych
\autocite{mahesh_2019_ml}. Uczenie maszynowe jest wykorzystywane do
uczenia maszyn efektywnego przetwarzania danych
\autocite{sindayigaya_2022_ml}. Algorytmy uczenia maszynowego można
podzielić na cztery główne podejścia: uczenie nienadzorowane (ang.
\emph{unsupervised learning}), uczenie nadzorowane (ang.
\emph{supervised learning}), uczenie częściowo nadzorowane (ang.
\emph{semi-supervised learning}) oraz uczenie przez wzmacnianie (uczenie
posiłkowane, ang. \emph{reinforcement learning})
\autocite{sarker_2021_ml}.

W ciągu ostaniach dwudziestu lat zaproponowano kilka różnych algorytmów
uczenia maszynowego do klasyfikacji obrazów satelitarnych
\autocite{sheykhmousa_2020_svm_vs_rf}, zazwyczaj wykorzystujące techniki
klasyfikacji bez nadzoru i klasyfikacji nadzorowanej
\autocite{ismail_2009_classification}.

Uczenie nienadzorowane analizuje nieoznakowane zbiory danych bez
konieczności ingerencji człowieka. Uczenie bez nadzoru jest powszechnie
stosowane do eksploracji danych, ekstrakcji cech generatywnych,
identyfikacji istotnych trendów i struktur oraz grupowania wyników. Ta
technika uczenia maszynowego jest najczęściej używana do grupowania
(klastowania), redukcji wielowymiarowości (redukcji cech) oraz
identyfikacji skojarzeń i relacji \autocite{sarker_2021_ml}.

Nadzorowane algorytmy uczenia maszynowego wykorzystują oznaczone dane
treningowe do znajdywania powiązań pomiędzy różnymi zmiennymi. Proces
uczenia nadzorowanego zachodzi, gdy określone cele mają zostać
osiągnięte na podstawie konkretnego zestawu danych wejściowych
(treningowych). Dwa główne typy uczenia nadzorowanego to klasyfikacja,
która separuje dane, oraz regresja, która dopasowuje dane
\autocite{sarker_2021_ml}.

W badaniu do klasyfikacji wykorzystano nadzorowaną metodę lasów losowych
(ang. \emph{Random Forest}, RF) \autocite{breiman_2001_rf}.

\hypertarget{sec-random-forest}{%
\subsection{Metoda lasów losowych}\label{sec-random-forest}}

Random Forest stał się jednym z najpopularniejszych klasyfikatorów
uczenia maszynowego wykorzystywanych przez społeczność teledetekcyjną ze
względu na dokładność jego klasyfikacji oraz wysoką wydajność
\autocite{belgiu_2016_rf,sheykhmousa_2020_svm_vs_rf}. Metoda lasów
losowych nie jest wrażliwa na szumy ani przetrenowanie, ponieważ nie
opiera się na ważeniu \autocite{gislason_2006_rf}. Dodatkowo lasy losowe
są znacznie lżejsze obliczeniowo niż metody oparte na wzmacnianiu (ang.
\emph{boosting}) \autocite{gislason_2006_rf} oraz mogą przetwarzać różne
typy danych, w tym zobrazowania satelitarne i dane numeryczne
\autocite{talukdar_2020_lulc}.

Random Forest (RF) to algorytm uczenia maszynowego, będący rozszerzeniem
idei drzew decyzyjnych \autocite{hejmanowska_2020_dane}. Random Forest
jest klasyfikatorem zespołowym (ang. \emph{ensemble classifier}), co
oznacza, że wykorzystuje dużą liczbę drzew decyzyjnych w celu
przezwyciężenia słabości pojedynczego drzewa \autocite{aaron_2018_ml}.

\ldots{} do opisania sposób działania (ciężkie)\ldots{}

\hypertarget{oprogramowanie}{%
\section{Oprogramowanie}\label{oprogramowanie}}

\hypertarget{qgis}{%
\subsection{QGIS}\label{qgis}}

do wektoryzacji -- tworzenia danych treningowych/testowych

\hypertarget{snap-i-sentinel-1-toolbox}{%
\subsection{SNAP i Sentinel-1 Toolbox}\label{snap-i-sentinel-1-toolbox}}

SNAP \autocite{snap}, czyli Sentinel Application Platform to platforma
oprogramowania rozwijana wspólnie przez firmy Brockmann Consult,
SkyWatch i C-S na zlecenie Europejskiej Agencji Kosmicznej (ESA),
przeznaczona do naukowego wykorzystania misji optycznych i mikrofalowych
Sentinel \autocite{snap-desktop,esa_snap}. Oprogramowanie SNAP zawiera
zestawy narzędzi do wizualizacji, przetwarzania oraz analizy danych
teledetekcyjnych, a zaimplementowane narzędzie do przetwarzania grafów
(ang. \emph{Graph Processing Tool}, GPT) daje możliwość tworzenia
łańcuchów procesów przetwarzania danych zdefiniowanych przez użytkownika
\autocite{hejmanowska_2020_dane,moskolai_2022_s1_workflow}. Struktura
przetwarzania grafów (ang. \emph{Graph Processing Framework}, GPF) w
oprogramowaniu SNAP służy do wsadowego przetwarzania danych za
pośrednictwem języka Extensible Markup Language (XML)
\autocite{moskolai_2022_s1_workflow}.

Przetwarzanie danych pochodzących z misji Sentinel-1 umożliwia zestaw
narzędzi S1TBX \autocite{s1tbx}, przeznaczony do przetwarzania danych
radarowych. Zestaw narzędzi Sentinel-1 Toolbox (S1TBX) zawiera narzędzia
do kalibracji, filtrowania plamek (tzw. efektu pieprzu i soli),
koregistracji, ortorektyfikacji, mozaikowania, konwersji danych,
polarymetrii i interferometrii \autocite{sentinel-1-toolbox}. Sentinel-1
Toolbox jest opracowywany dla ESA przez firmę Array we współpracy z DLR,
Brockmann Consult i OceanDataLab \autocite{sentinel-1-toolbox}.

\hypertarget{ux15brodowisko-jux119zyka-r}{%
\subsection{Środowisko języka R}\label{ux15brodowisko-jux119zyka-r}}

Czynności związane z końcowym przygotowaniem danych wejściowych oraz
bezpośrednio z uczeniem maszynowym zostały wykonane z wykorzystaniem
środowiska języka R \autocite{R-base}. R to wieloplatformowy język
programowania o otwartym kodzie źródłowym do obliczeń statystycznych i
wizualizacji danych \autocite{lovelace_2019_geocomputation}. Dzięki
dużej liczbie pakietów R obsługuje również statystki geoprzestrzenne,
modelowanie oraz wizualizację danych przestrzennych
\autocite{lovelace_2019_geocomputation}. W pracy wykorzystane zostało
zintegrowane środowisko programistyczne (ang. \emph{Integrated
Development Environment}, IDE) RStudio
\autocite{rstudio_team_2020_rstudio} przeznaczone dla języka R. Poza
standardowymi możliwościami środowiska R, w procesie pracy wykorzystane
zostały pakiety stworzone przez społeczność R w celu rozszerzenia
funkcjonalności tego języka. Do operacji na danych rastrowych
zastosowano pakiet \emph{terra} \autocite{R-terra}, natomiast do
przetwarzania danych wektorowych używany był pakiet \emph{sf}
\autocite{R-sf}. Losowe generowanie danych przestrzennych umożliwia
pakiet \emph{spatstat.random} \autocite{R-spatstat.random} z rodziny
pakietów \emph{spatstat} \autocite{R-spatstat}. Do przeprowadzenia
analizy oraz predykcji opartej o elementy uczenia maszynowego
wykorzystano pakiet \emph{mlr3} \autocite{R-mlr3}, w ramach którego
użyty został algorytm lasów losowych zaimplementowany w pakiecie
\emph{ranger} \autocite{R-ranger}. Do obliczeń związanych z uczeniem
maszynowym wykorzystano pakiet \emph{future} \autocite{R-future},
umożliwiający równoległe (wielowątkowe) przetwarzanie wyrażeń R,
pozwalające na skrócenie czasu realizacji zadań w stosunku do
przetwarzania sekwencyjnego.

?GLCMTextres jeśli dojdą tekstury obrazu

\bookmarksetup{startatroot}

\hypertarget{sec-wyniki}{%
\chapter{Wyniki}\label{sec-wyniki}}

Część \textbf{Wyniki} może składać się z jednego lub więcej rozdziałów.
Każdy z tych rozdziałów powinien mieć tytuł adekwatny do swojej treści.

Rozdziały wynikowe powinny korzystać z wiedzy opisanej w poprzednich
rozdziałach (Rozdziały \ref{sec-lit}, \ref{sec-materialy},
\ref{sec-metody}). W przypadku prac analitycznych, ich treść powinna
przedstawiać kolejne etapy eksploracji i analizy danych. W przypadku
prac technicznych, treść tych rozdziałów powinna opisywać stworzone
narzędzia, a następnie pokazywać ich zastosowanie/a.

W przypadku prac technicznych warto pokazywać fragmenty napisanego
rozwiązania lub jego wywołania używając bloków kodu.

\begin{Shaded}
\begin{Highlighting}[]
\NormalTok{moja\_funkcja }\OtherTok{=} \ControlFlowTok{function}\NormalTok{(x)\{}
  \FunctionTok{cat}\NormalTok{(x, }\StringTok{"rządzi!"}\NormalTok{)}
\NormalTok{\}}
\FunctionTok{moja\_funkcja}\NormalTok{(}\StringTok{"Autor tej pracy"}\NormalTok{)}
\end{Highlighting}
\end{Shaded}

\begin{verbatim}
Autor tej pracy rządzi!
\end{verbatim}

\bookmarksetup{startatroot}

\hypertarget{podsumowanie}{%
\chapter{Podsumowanie}\label{podsumowanie}}

Podsumowanie pracy jest w pewnym sensie znacznie rozbudowanym
abstraktem. Należy wyliczyć i opisać osiągnięcia uzyskane w pracy
dyplomowej. Tutaj jednak (w przeciwieństwie do np. rozdziału
\ref{sec-wprowadzenie}) należy przechodzić od szczegółu do ogółu - co
zostało stworzone/określone, jak zostało to zrobione, jakie ma to
konsekwencje, itd.

Ten rozdział powinien też zawierać opis kwestii, których nie udało się
rozwiązać w pracy dyplomowej (i dlaczego się nie udało) oraz pomysły na
przyszłe ulepszenie uzyskanych wyników lub dalsze badania.

\printbibliography[heading=bibintoc, title=Bibliografia]

\end{document}
